\documentclass[aspectratio=169,t,11pt,table]{beamer}
\usepackage{../../slides,../../math}

% Optionally define `accent`/`accent2` colors for theme customization
% I recommend changing the top slider on this: https://hslpicker.com/#1e9400
\definecolor{accent}{HTML}{2B5269}
\definecolor{accent2}{HTML}{9D2235}

\title{Course Introduction}
\subtitle{\it  ECON 5783 — University of Arkansas}
\date{Fall 2024}
\author{Prof. Kyle Butts}

\begin{document}

% ------------------------------------------------------------------------------
\begin{frame}[noframenumbering,plain]
\maketitle

% \bottomleft{\footnotesize $^*$A bit of extra info here. Add an asterich to title or author}
\end{frame}
% ------------------------------------------------------------------------------

\begin{frame}{}
  \bigskip
  \begin{center}
    \alert{\Large\bf Welcome to Applied Microeconometrics (Econ 5783)}
  \end{center}

  \bigskip
  My name is Kyle Butts, Ph.D.
  \begin{itemize}
    \item Please call me Kyle
    
    \item Graduated from University of Colorado Boulder
  \end{itemize}  

  \bigskip
  Research: 
  \begin{itemize}
    \item Econometrics and Causal Inference
    \begin{itemize}
      \item Panel data methods, spillover effects
    \end{itemize}
    \item Urban economics 
    \begin{itemize}
      \item Zoning laws, landlords, place-based policies
    \end{itemize}
  \end{itemize}
\end{frame}

\begin{frame}{Norms and Background}
  All \alert{\bf questions are appreciated}. 
  \begin{itemize}
    \item I \emph{always} have time to answer them.
  \end{itemize}

  \pause
  \bigskip
  This course will assume you have experience in basic graduate level econometrics courses
  \begin{itemize}
    \item E.g. you have experience with linear regression, hypothesis testing, etc. 
  \end{itemize}
\end{frame}

\begin{frame}{Course Materials}{Coding Software}
  You will need to download \emph{two} programs:
  \begin{enumerate}
    \item Install \texttt{R} from \url{https://cloud.r-project.org/}.
    \item Install \texttt{Positron} from \url{https://github.com/posit-dev/positron/releases}. 
  \end{enumerate}

  \bigskip
  If you do not have experience in \texttt{R}, then I encourage you to spend a few hours learning basics of loading and working with data
  \begin{itemize}
    \item \url{https://github.com/cobriant/tidyverse_koans} and \url{https://r4ds.hadley.nz/} are good resources for learning \texttt{R}
    \item If you have Stata experience, you can use \url{https://stata2r.github.io/} to help transition
  \end{itemize}
\end{frame}

\begin{frame}{Course Materials}{Textbooks}
  Our primary text for the course is \href{https://mixtape.scunning.com/}{Causal Inference: The Mixtape} by Scott Cunningham. 
  \begin{itemize}
    \item Available free online
  \end{itemize}
  
  \bigskip
  Will use review articles for overviews of methodologies and academic articles for examples of empirical usage
  \begin{itemize}
    \item In the \texttt{`Literature/`} folder in the GitHub repository
  \end{itemize} 
\end{frame}

\begin{frame}{Responsible Researchers}
  I have tried to incorporate practical advice on doing applied research throughout this class. 
  
  \bigskip
  In particular, I try to teach you about how to \emph{properly talk about causal research},  how to clearly lay out assumptions and how to assess their plausibility, and how to avoid saying incorrect things
  \begin{itemize}
    \item E.g. you can never `validate' or `prove' an assumption. \alert{Do not write this}
  \end{itemize}
\end{frame}

\begin{frame}{Assignments}{Problem Sets}
  Problems sets will be assigned for each topic we cover in the course
  \begin{itemize}
    \item Will require you to analyze datasets, run Monte Carlo simulations, and answer interpretative questions
    \item Solutions in \texttt{R}, but you can use whatever language you want
  \end{itemize}

  \bigskip
  Students are encouraged to work in groups to discuss how the approach the problem sets, but each student must hand in his or her own set of answers. 
\end{frame}

\begin{frame}{Assignments}{Exams}
  Two exams on October 7th and November 13th, but subject to change +/- a class

  \begin{itemize}
    \item Each exam covers half of the course material
    
    \item Will ask you to think about novel examples and think about causal assumptions
  \end{itemize}
\end{frame}

\begin{frame}{Assignments}{Final Project}
  You will have a final project that serves as your `final'.
  \begin{itemize}
    \item Answer a research question in the form of "the impact of X on Y"
  \end{itemize}  

  \bigskip
  You will be graded on your ability to discuss the assumptions needed to establish causality and \textbf{not whether or not you have a flawless research design}
  \begin{itemize}
    \item Non-PhD Students may work in a group of two and (ii) are recommended but not required to include a brief literature review

    \item PhD Students should submit a single-authored paper and include a brief literature review
  \end{itemize}
\end{frame}


\end{document}
