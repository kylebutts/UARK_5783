\documentclass[aspectratio=169,t,11pt,table]{beamer}
\usepackage{../../slides,../../math}
\definecolor{accent}{HTML}{2B5269}
\definecolor{accent2}{HTML}{9D2235}

\title{Topic 4: Regression Discontinuity}
\subtitle{\it  ECON 5783 — University of Arkansas}
\date{Fall 2024}
\author{Prof. Kyle Butts}

\begin{document}

% -----------------------------------------------------------------------------
\begin{frame}[noframenumbering,plain]
\maketitle

% \bottomleft{\footnotesize $^*$A bit of extra info here. Add an asterich to title or author}
\end{frame}
% -----------------------------------------------------------------------------

\begin{frame}{Regression Discontinuity Design (RDD)}{Example 1}
  Lee (2008, JOE) sutdies the ``incumbency advantage'', the hypothesis that being a serving elected official improves future election outcomes

  \pause
  \bigskip
  The problem, of course, is that candidates who won their election usually are different than those that do not 
  \begin{itemize}
    \item e.g. are more charming, in a more one-sided district, have a better resume
  \end{itemize}

  \pause
  \bigskip
  Lee's idea is to compare candidates who narrowly lost to those that narrowly won 
  \begin{itemize}
    \item By having similar vote percentage, Lee hopes, to be comparing candidates with similar unobservables
  \end{itemize}
\end{frame}

\imageframe{figures/lee_win_tp1.pdf}

\begin{frame}{Regression Discontinuity Design (RDD)}{Example 1}
  There is a clear jump from candidates that narrowly lost to candidates that narrowly won
  \begin{itemize}
    \item The main concern is that candidates that just lost look different in terms of unobservables to those that just won
  \end{itemize}

  \pause
  \bigskip
  To try and alleviate this concern, Lee checks if other observed variables `jump' at the cutoff 
  \begin{itemize}
    \item A jump in other `pre-determined' variables at the cutoff would be problematic to our story
  \end{itemize}
\end{frame}

\imageframe{figures/lee_win_prior_office_exp.pdf}

\begin{frame}{Regression Discontinuity Design (RDD) Terminology}{}
  In the RDD literature, we will have a \alert{`running' variable} (`score' variable), $X_i$, and a \alert{`cutoff'} value $c$
  \begin{itemize}
    \item Units with $X_i < c$ do not receive the `policy` and units with $X_i 
    \geq c$ do
  \end{itemize}

  \bigskip 
  The treatment variable is defined as $D_i = \one{X_i \geq c}$
\end{frame}

\begin{frame}{Regression Discontinuity Design (RDD)}{Example 2}
  Bleemer and Mehta (2022, AEJ Applied) study the returns to being an economics major by leveraging a requirement of a 2.8 GPA threshhold in Econ 1 and 2 at UCSC:

  \begin{itemize}
    \item Comparing students just above a 2.8 GPA to those just below helps address selection into economics major
    
    \item A potential concern is that highly-motivated students near the threshold might ask for higher grade, extra credit, etc.
  \end{itemize}
\end{frame}

\begin{frame}{}
  \begin{columns}[T]
    \begin{column}{.5\textwidth}\vspace*{-\bigskipamount}
      \includegraphics[width = \textwidth]{figures/bleemer_mehta_1.png}
    \end{column}
    \begin{column}{.5\textwidth}\vspace*{-\bigskipamount}
      \includegraphics[width = \textwidth]{figures/bleemer_mehta_2.png}
    \end{column}
  \end{columns}
\end{frame}

\begin{frame}{Regression Discontinuity Design (RDD)}{Example 3}
  Hoekstra (2009, RESTAT) estimates the impact of attending a `flagship university' on future earnings
  \begin{itemize}
    \item There was an internal (secret) SAT cutoff that had a large increase in the probability of acceptance
  \end{itemize}

  \bigskip
  That this cutoff was secret helps with concerns over retaking SAT, e.g.  
\end{frame}

\imageframe{figures/hoekstra_2009_1.png}
\imageframe{figures/hoekstra_2009_2.png}

\begin{frame}{Regression Discontinuity Design (RDD)}{Example 4}
  Angrist and Lavy (1999, QJE) study the effect of class size on kid's learning.
  They use a rule in Israel that requires all classes to have 40 or fewer students
  \begin{itemize}
    \item This creates sharp drops in class size at 41, 81, 121, 161 students
  \end{itemize}
\end{frame}

\imageframe{figures/angrist_lavy_1999.png}


\begin{frame}{Regression Discontinuity Design (RDD)}{Example 5}
  Anderson and Magruder (2011, Economics Journal) study the impact of restaurant Yelp ratings on restaurant business outcomes
  \begin{itemize}
    \item Yelp takes the average rating (e.g. 3.27) and shows the nearest number of stars rounding/up or down. E.g. 3.24 rounds down to 3 and 3.25 rounds up to 3.5
  \end{itemize}

  \pause
  \bigskip
  Their argument is that the score, average rating. is a noisy measure of the true quality and so restaurants look the same on either side of the cutoff
  \begin{itemize}
    \item Their main concern is `manipulation' of running variable (we'll come back to this)
  \end{itemize}
\end{frame}

\begin{frame}{Regression Discontinuity Design (RDD)}{Example 6}
  Turner et. al. (2014, ECTA) study the impacts of land-use regulations on the value of land
  \begin{itemize}
    \item They compare homes on one side of a zoning border to homes on the others
    \item The assumption is that the neighborhood doesn't `abruptly' change when crossing the zoning boundary
  \end{itemize}

  \bigskip 
  This is an example of a spatial RDD which we'll discuss more about later
\end{frame}

\begin{frame}{Difficulties with RDD}
  We can not do our usual strategy of comparing treated and untreated individuals with the same $X_i$
  \begin{itemize}
    \item Either everyone is treated or no one is
  \end{itemize}

  \bigskip
  But, there is one point where we \emph{kind of} have treated and untreated units: the cutoff $c$
  \begin{itemize}
    \item This is our intuition for using `just above' versus `just below' the cutoff.
  \end{itemize}
\end{frame}

\section{Formalizing RDD Identification Arguments}

\begin{frame}{Formalizing RDD (Take 1)}{``Noisy'' running variable}
  Let's use a canonical example of students taking a test ($X_i$) and students above a certain cutoff, $c$, are treated (e.g. put in honor's class).

  \bigskip
  Students will vary in terms of their expected score, $\sigma_i = \expec{X_i}$. We think that future outcomes $Y_i$ vary systematically based on $\sigma_i$
  \begin{itemize}
    \item Comparing students above and below the cutoff will be biased  
  \end{itemize}
\end{frame}

\begin{frame}{``Noisy'' running variable creates an experiment}{}
  For random reasons (a weird question, skipped breakfast, etc.) students score is given by $X_i = \sigma_i + \varepsilon_i$, where $\varepsilon$ is random noise.
  \begin{itemize}
    \item For students with $\sigma_i$ close to $c$, $\varepsilon_i$ will make them above or below the cutoff
  \end{itemize}

  \pause
  \bigskip
  This means for $\sigma_i$ close to $c$, we have a \emph{natural experiment}
  \begin{itemize}
    \item Of course, we don't observe $\sigma_i$, so instead we take students within some range $c \pm h$. 
    
    \item If $h$ is ``small enough'', we can do a difference in means between students below and above the cutoff
  \end{itemize}
\end{frame}

\imageframe{figures/plot_dgp_1_rand_tau.pdf}

\begin{frame}{``Local'' treatment effect}{}
  In essence, we throw out all the data outside of $\left[ c-h, c+h \right]$ and compare students with $\sigma_i \approx c$
  
  \bigskip
  This means we only learn about what the impact of treatment is for students with $\sigma_i \approx c$
  \begin{itemize}
    \item E.g. if the cutoff is an SAT score of 1950, then we estimate the impact of treatment for students with around that score
  \end{itemize}
\end{frame}

\begin{frame}{``Local'' treatment effect}{}
  That is, we compare the following:
  \begin{align*}
    &\expec{Y_i}{c < \sigma_i < c+h} - \expec{Y_i}{c-h < \sigma_i < c} \\
    &\quad = \expec{Y_i(1)}{c < \sigma_i < c+h} - \expec{Y_i(0)}{c-h < \sigma_i < c} \\
    \pause
    &\quad \approx \expec{Y_i(1)}{\sigma_i = c} - \expec{Y_i(0)}{\sigma_i = c}
  \end{align*}

  That is, we estimate the CATE for people with $\sigma_i \approx c$
\end{frame}

\begin{frame}{How much `noise' is there in the running variable?}{}
  How do we know how large to make the cut-off $h$? 
  \begin{itemize}
    \item The gold-star answer is to have some application-specific understanding to know how much noise there is and take $h$ to be half that noise
  \end{itemize}

  \bigskip
  Otherwise, we are left to trying to determine a `reasonable' $h$
\end{frame}

\begin{frame}{How much `noise' is there in the running variable?}{}
  There is a trade-off between using a smaller or larger $h$
  \begin{itemize}
    \item On the one hand, a smaller $h$ makes it more likely that units in $(c-h, c)$ look similar to units in $(c, c+h)$
    
    \item On the other, a larger $h$ uses more observations for estimation. A larger $h$ runs a risk of including units that differ systematically from the $\sigma_i = c$ units
  \end{itemize}

  When data on attributes of units are available, then we can use those to help with determining $h$
\end{frame}

\begin{frame}{Estimation of $h$ using extra covariates}{}
  Cattaneo, Frandsen and Titiunik (2015, Journal of Causal Inference) recommend a procedure (available in \texttt{rdlocrand}) where:
  \begin{itemize}
    \item Start with a very small $h_1$ and test if the mean of $\bm{X}_i$ are the same in $(c - h_1, c)$ and $(c, c + h_1)$
    
    \item If you fail to reject the null of no difference in means, then expand to $h_2$
    
    \item Continue until you reject the null
  \end{itemize}

  \bigskip
  The idea being, select the largest $h$ where units `look the same' on both sides of the cutoff.
\end{frame}


\begin{frame}[fragile]{\texttt{rdlocrand} basic syntax}
  \vspace*{-\medskipamount}
  \begin{codeblock}
library(rdlocrand)

# Estimate effect and get p-values
rdrandinf(
  Y = df$y, R = df$score, cutoff = 0, wl = 0.025, wr = 0.025
)

# Estimate optimal h
rdwinselect(
  R = df$score, X = cbind(df$x1, df$x2, df$x3), obsmin = 10, wobs = 5
)
  \end{codeblock}  
\end{frame} 

\begin{frame}{Local randomization argument}
  This approach has a nice intuition: local to the cutoff we have a quasi-experiment due to noise in the running variable

  \bigskip
  Despite this, this approach is less popular than the next approach we discuss
  \begin{itemize}
    \item In part, sometimes we don't believe there is noise in the running variable
    
    \item The plots, like in Lee (2008), are the bread and butter of the RDD
    \begin{itemize}
      \item These are not "difference-in-means" but instead rely on trying to identify a "jump" in smoothed lines
    \end{itemize} 
  \end{itemize}
\end{frame}

\begin{frame}{Formalizing RDD (Take 2)}{Continuity of outcomes}
  The second approach we will discuss is the older (and more commonly used) approach to RDD estimation
  \begin{itemize}
    \item Instead of assumping the score is randomly assigned, this method will rely on a `continuity assumption' of potential outcomes
  \end{itemize}

  \pause
  \bigskip
  The \alert{continuity assumption} says that both the treated and untreated potential outcomes evolve `smoothly' and do not have an abrupt `jump' (discontinuity) at the cutoff
  \begin{itemize}
    \item This is what our mind `tells us to do' when we see these RDD plots
  \end{itemize}
\end{frame}

\imageframe{figures/lee_win_tp1.pdf}

\begin{frame}{Formalizing continuity}
  The goal of our treatment effect estimator is to identify the following:
  $$
    \expec{Y_i(0)}{X_i = c} \ \text{ and }\ \expec{Y_i(1)}{X_i = c}
  $$

  We don't typically observe anyone with $X_i$ exactly equal to 0 (assuming continuous $X_i$)
  \begin{itemize}
    \item So, necessarily we need to extrapolate to the cutoff using observations away from the cutoff
  \end{itemize}
\end{frame}

\begin{frame}{Formalizing continuity}
  Simliar to the selection on observables, define:
  $$
    \mu_d(x) = \expec{Y_i(d)}{X_i = x}
  $$
  to be the conditional expectation of $Y_i(0)/Y_i(1)$ conditional on the running variable being equal to $x$

  \bigskip
  We will fit $\mu_0(x)$ using obseravtions with $X_i < c$ and $\mu_1(x)$ using $X_i > c$
  \begin{itemize}
    \item E.g. fit a linear model of $X_i$ with observations in $(c - h, c)$ / $(c, c+h)$
  \end{itemize}
\end{frame}

\begin{frame}{Extrapolation as limits}
  We are able to learn about the relationship between $Y_i$ and $X_i$ away from the cutoff. We are going to need to take out model and \alert{extrapolate} it to the cutoff 
  \begin{itemize}
    \item Imagine taking averages of $Y_i$ from $(c, c+h)$ to estimate $\expec{Y_i(1)}{X_i = c}$. Assuming infinite data, as $h \to 0$ our average should get closer to closer to the true CEF.
  \end{itemize}
  
  \bigskip
  More formally, under `continuity' we have
  \begin{align*}
    \expec{Y_i(1)}{X_i = c} 
    &= \lim_{s \downarrow c} \expec{Y_i(1)}{X_i = x}
  \end{align*}
  \begin{itemize}
    \item Note here, we are taking the limit from above to only use obs. with $Y_i = Y_i(1)$
  \end{itemize}
\end{frame}

\begin{frame}{RDD Estimand}
  The \alert{regression discontinuity} estimand is formed as follows:
  \begin{align*}
    \tau_{\texttt{RD}} &= 
    \expec{Y_i(1) - Y_i(0)}{X_i = c} \\
    &= \lim_{s\downarrow c} \expec{Y_i}{X_i = x} - \lim_{s \uparrow c} \expec{Y_i}{X_i = x} 
  \end{align*}

  \begin{itemize}
    \item In words, take the difference between the right-hand limit of $\mu_1(x)$ and the left-hand limit of $\mu_0(x)$
  \end{itemize}
\end{frame}

\begin{frame}{Why `continuity'?}
  Since we are fitting a model using observations away from the cutoff $c$, we are \alert{extrapolating} the estimated $\mu_0(x)$ from the range of $X_i$ we used for estimation to the cutoff $X_i = c$
  \begin{itemize}
    \item For this to work, we need the $\mu_d(x)$ to be continuous in a neighborhood around $c$
  \end{itemize}
\end{frame}

\begin{frame}{Why might continuity fail?}{}
  One of the main concerns people have with an RDD empirical application is that units are `sorting' near the cutoff:
  \begin{itemize}
    \item E.g. if there is an SAT cutoff for a scholarship, some kinds of students who were close to making the threshold might retake the SAT multiple times
  \end{itemize}

  \bigskip
  If students who retook look different than those that don't, $\implies$ at a jump in $Y_{i}(0)$ at the cutoff
  \begin{itemize}
    \item $c + \varepsilon$ is `contaminated' by the retakers 
  \end{itemize}
\end{frame}

\begin{frame}{Why might continuity fail?}{}
  In general, the main intuition is that we want there to be no `sorting' around the threshold
  \begin{itemize}
    \item If we think units' characteristics (observable and unobservable) are smooth over the cutoff, then it's reasonable to assume the potential outcomes are smooth too
  \end{itemize}

  \pause
  \bigskip
  For the covariates you do observe, can show that these do not jump at the cutoff
  \begin{itemize}
    \item Hopefully, the unobservables do not as well
  \end{itemize}
\end{frame}

\begin{frame}{Using observations `near' the cutoff}{}
  Returning to the notion of continuity, to estimate $\mu_d(x)$ at the cutoff $X_i = c$, we want to use observations very close to the cutoff
  \begin{itemize}
    \item So that we extrapolate as little as possible
  \end{itemize}

  \bigskip
  In finite samples, there might be very few observations near the cutoff
  \begin{itemize}
    \item In these settings, noise in the data can make it hard to learn information about the data-generating process
  \end{itemize}
\end{frame}

\begin{frame}{Estimation of RDD}
  There are \emph{many} different approaches to estimation of the RD coefficient, but there are a few common questions:
  \begin{enumerate}
    \item How large of a window around the cutoff should we use?
    \begin{itemize}
      \item Smaller window relies on less `extrapolation' but is `noisier' (bias-variance trade-off)
    \end{itemize}
    
    \item How should we fit the model of $\expec{Y_i(d)}{X_i = x}$
    \begin{itemize}
      \item Simpler models are more robust to extrapolation, but may get the functional form wrong
    \end{itemize}
  \end{enumerate}
\end{frame}

\imageframe{figures/mhe_ex_linear.pdf}
\imageframe{figures/mhe_ex_nonlinear.pdf}
\imageframe{figures/mhe_ex_mistake.pdf}


\begin{frame}{Difference-in-means estimation}{}
  For now, take the window to be $(c-h, c+h)$; we will return to the choice of $h$ later in the slides

  \bigskip
  The simplest estimator is the \emph{locally constant estimator}
  $$
    \mu_0(x) = \expechat{Y_i}{X_i \in (c-h, c)} \ \text{ and }\ \mu_1(x) = \expechat{Y_i}{X_i \in (c-h, c)}
  $$
  \begin{itemize}
    \item This is our difference-in-means estimator
  \end{itemize}
\end{frame}

\begin{frame}{Estimation via Regression}{}
  This can be estimated with the simple regression:
  $$
    Y_i = \alpha_0 + \alpha_1 D_i + u_i
  $$
  on the subsample with $X_i \in (c-h, c+h)$
\end{frame}

\imageframe{figures/plot_dgp_2_tau.pdf}
\imageframe{figures/plot_dgp_2_local_constant_estimator.pdf}

\begin{frame}{Locally-linear estimation}{}
  Now, let's use a linear-model $\mu_d(x) = \alpha_d + \beta_d X_i$
  \begin{itemize}
    \item Note we let the slope vary for $Y_i(0)$ and $Y_i(1)$
  \end{itemize}

  \bigskip
  Take our estimated models and form the regression adjustment estimator as:
  \begin{align*}
    \hat{\tau} 
    &= \left( \hat{\alpha}_1 + \hat{\beta}_1 c \right) - \left( \hat{\alpha}_0 + \hat{\beta}_0 c \right)
  \end{align*}
  \begin{itemize}
    \item This looks like our regression adjustment estimator!
  \end{itemize}
\end{frame}

\begin{frame}{Estimation via Regression}{}
  The locally linear estimator can be estimated with an interacted regression (like with regression adjustment):
  $$
    Y_i = \alpha_0 + \alpha_1 D_i + \beta_0 (X_i - c) + \beta_1 D_i (X_i - c) + u_i
  $$
  on the subsample with $X_i \in (c-h, c+h)$

  \begin{itemize}
    \item Note we recenter, $X_i - c$, so that $\hat{alpha}_1$ is the RDD estimate (e.g. "margin of victory" instead of "vote-share" in Lee, 2008)
  \end{itemize}
\end{frame}

\imageframe{figures/plot_dgp_2_local_linear_estimator.pdf}

\begin{frame}{Why linear if our data looks wiggly?}
  There is a lingering question, if we think our data is very `wiggly' why are we assuming linearity in $X_i$?

  \begin{itemize}
    \item Remember we are using only observations \emph{local} to the cutoff, so (at least asymptotically) wiggly functions are approximately linear on small bandwidths
    
    \item Essentially we are using the logic of the Taylor expansion that we can approximate a function locally using a lienar function
  \end{itemize}
\end{frame}


\begin{frame}{Different Bandwidths}{}
  Our estimator, implicitly depends on our choice of bandwidth $h$
  \begin{itemize}
    \item Larger $h$ uses more observations so should help with precision of our estimate
    \item but relies more on functional form for extrapolation (Taylor approximation only holds locally)
  \end{itemize}
\end{frame}

\imageframe{figures/plot_dgp_2_local_constant_large_bw.pdf}
\imageframe{figures/plot_dgp_2_local_linear_large_bw.pdf}

\begin{frame}{Local polynomial estimation}{}
  We can extend our logic to higher-order polynomials:
  $$
    Y_i = \alpha_0 + \alpha_1 D_i \sum_{p = 1}^\rho \beta_{0,p} (X_i - c)^p + \sum_{p = 1}^\rho \beta_{1,p} (X_i - c)^p + u_i
  $$
  
  Still, $\hat{\alpha_1}$ is our RDD estimator
\end{frame}

\imageframe{figures/plot_dgp_2_local_quadratic_estimator.pdf}

\begin{frame}{RDD By Hand}{}
  The most `straight forward' way to estimate the RDD is to do two regressions:
  \begin{enumerate}
    \item Regress $Y_i$ on a $k$-th order polynomial of $X_i - c$ for $c - h < X_i < c$
    \item Regress $Y_i$ on a $k$-th order polynomial of $X_i - c$ for $c < X_i < c + h$
  \end{enumerate}

  \bigskip
  Predict $Y$ at $X_i = 0$ for both models. These are your estimates for $\hat{\mu}^-$ and $\hat{\mu}^+$ respectively. Then our RD estimate can be formed as
  $$
    \hat{\tau}_{\texttt{RD}} = \hat{\mu}^+ - \hat{\mu}^-
  $$
\end{frame}


\begin{frame}{Example: Punishment and Deterrence: Evidence from Drunk Driving}
  Hansen (2015, AER) considers the impact of getting a DUI (driving while drunk) has on future drinking behavior
  \begin{itemize}
    \item Uses the `legal limit' of a blood-alcohol content of 0.08 as the RDD `cutoff'
  \end{itemize}
\end{frame}

\imageframe{figures/hansen_raw_avgs.pdf}

\begin{frame}[fragile]{RDD by hand: Randomization-based estimation}
	\begin{codeblock}
feols(
  recidivism ~ 1 + over_limit,
  data = subset(hansen, bac1 >= 0.07 & bac1 <= 0.09)
)
	\end{codeblock}
\end{frame}

\begin{frame}[fragile]{RDD by hand: Randomization-based estimation}
  \begin{codeblock}[{}]
OLS estimation, Dep. Var.: recidivism
Observations: 18,897
Standard-errors: IID 
                Estimate Std. Error  t value   Pr(>|t|)    
(Intercept)     0.114784   0.003253 35.28149  < 2.2e-16 ***
over_limitTRUE -0.017903   0.004472 -4.00349 6.2652e-05 ***
---
Signif. codes:  0 '***' 0.001 '**' 0.01 '*' 0.05 '.' 0.1 ' ' 1
  \end{codeblock}
\end{frame}

\begin{frame}[fragile]{RDD by hand: Locally-linear estimation}
  \vspace{-\bigskipamount}
	\begin{codeblock}
feols(
  recidivism ~ 1 + over_limit + 
    I(bac1 - 0.08) + I(bac1 - 0.08) * over_limit,
  data = subset(hansen, bac1 >= 0.04 & bac1 <= 0.12)
)
	\end{codeblock}
\end{frame}

\begin{frame}[fragile]{RDD by hand: Locally-linear estimation}
  \begin{codeblock}[{}]
OLS estimation, Dep. Var.: recidivism
Observations: 72,880
Standard-errors: IID 
                                Estimate Std. Error  t value   Pr(>|t|)    
(Intercept)                    0.113956   0.003718 30.65215  < 2.2e-16 ***
over_limitTRUE                -0.020381   0.004694 -4.34212 1.4131e-05 ***
I(bac1 - 0.08)                -0.312061   0.226113 -1.38011 1.6756e-01    
over_limitTRUE:I(bac1 - 0.08)  0.707713   0.254253  2.78350 5.3790e-03 ** 
---
Signif. codes:  0 '***' 0.001 '**' 0.01 '*' 0.05 '.' 0.1 ' ' 1
  \end{codeblock}
\end{frame}

\begin{frame}[fragile]{RDD plot using \texttt{rdplot} from \texttt{rdrobust}}{}
	\begin{codeblock}
library(rdrobust)
rdplot(y = hansen$recidivism, x = hansen$bac1, c = 0.08, h = 0.4)
	\end{codeblock}
\end{frame}

\imageframe{figures/hansen_rdplot_bw_0pt4.pdf}

\begin{frame}[fragile]{RDD using \texttt{rdrobust}}{}
  \begin{codeblock}
est <- rdrobust(
  y = hansen$recidivism, x = hansen$bac1, c = 0.08
)
summary(est)
  \end{codeblock}
\end{frame}

\begin{frame}[fragile]{RDD using \texttt{rdrobust}}{}
  \begin{codeblock}[{}]
... Details ...

=============================================================================
        Method     Coef. Std. Err.         z     P>|z|      [ 95% C.I. ]       
=============================================================================
  Conventional    -0.018     0.006    -3.083     0.002    [-0.030 , -0.007]    
        Robust         -         -    -2.452     0.014    [-0.031 , -0.003]    
=============================================================================
  \end{codeblock}
\end{frame}


\begin{frame}{Sorting based on the running variable}{}
  As we discussed before, our main threat to the continuity assumption is that units are `sorting' onto either side of the cutoff based on their observable characteristics

  There are two standard `gut checks' that folks will want to see
\end{frame}

\begin{frame}{Check 1: Balance}{}
  First, if we have any `pre-determined' covariates, we want see that there are no jumps in the average of covariates at the cutoff
  \begin{itemize}
    \item Rerun RDD estimator using $X_i$ as the outcome variable. Test $\hat{\tau}_{\texttt{RD}}= 0$
  \end{itemize}

  \bigskip
  Use covariates that you would likely see a jump in if there was sorting (e.g. a proxy for `motivation' for SAT example)
\end{frame}

\imageframe{figures/hansen_balance_check_male.pdf}
\imageframe{figures/hansen_balance_check_white.pdf}

\begin{frame}{Check 2: McCrary Density}{}
  If you don't have extra covariates, you can check for `continuous density' a la McCrary (2006, JOE)

  \bigskip
  The idea is that if we think there is no sorting, then we would expect to see no strong `jump' in the histogram of the running variable
  \begin{itemize}
    \item A jump at the discontinuity suggests that agents are sorting onto one side of the border (assuming the density would be smooth in the absence of the cutoff policy)
  \end{itemize}
\end{frame}

\imageframe{figures/hansen_density_bac1.pdf}

\begin{frame}{Example}{Camacho and Conover (2011, AEJ: EP)}
  Camacho and Conover (2011, AEJ: EP) discuss a real example of `manipulation'
  \begin{itemize}
    \item Towns in Colombia receive social programs if their poverty index score is below a cutoff
     
    \item The poverty index algorithm becomes public in 1997
  \end{itemize}
\end{frame}

\imageframe{figures/camacho_conover_2011.png}

\begin{frame}{Example}{Camacho and Conover (2011, AEJ: EP)}
  The authors show in their paper that the towns that corruptly gamed their score looked different than those that did not
  \begin{itemize}
    \item Those that gamed their scores had more political competition
  \end{itemize}

  \bigskip
  In this setting, if we saw a big jump at the poverty index cutoff, then we can't know if it's from the social programs or from the higher-level of political competition 
\end{frame}

% TODO: Discuss "optimal" bandwidth selection
\begin{frame}{Optimal Bandwidth selection}
  As we discussed before, there is a tension with selection of bandwidth:
  \begin{itemize}
    \item Smaller bandwidths use data closer to the cutoff and rely less on functional form
    
    \item Larger bandwidths lets you use more data to estimate RD
  \end{itemize}

  All identification relies on the bandwidth getting smaller as the sample size grows
  \begin{itemize}
    \item But we are left wondering what the `optimal bandwidth' is given a finite sample
  \end{itemize}
\end{frame}


\begin{frame}{Optimal Bandwidth selection}
  Calonico, Cattaneo and Farrell (2018, Economics Journal) use statistical methods to identify the `optimal' bandwidth. The optimal bandwith is based on the following:
  \begin{itemize}
    \item Of course, the bandwidth should shrink as the sample size grows

    \item If the curvature of $\expec{Y_i}{X_i = x}$ is more wiggly, the bandwidth should be smaller

    \item If the variance of $Y_i$ near the cutoff is larger, then you should use a larger bandwidth
  \end{itemize}

  \pause
  \bigskip 
  The upshot is this is done automatically for you with \texttt{rdrobust}
  \begin{itemize}
    \item See Cattaneo et. al.'s review article for more details
  \end{itemize}
\end{frame}

% TODO: Discuss discrete ?


\section{Spatial RDD}
\begin{frame}{Spatial RDD}{}

\end{frame}

\begin{frame}{Spatial RDD}{Complications}

\end{frame}

\begin{frame}{A lot of things change at a border}{}

\end{frame}


\end{document}
    
