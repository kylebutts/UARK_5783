\documentclass[aspectratio=169,t,11pt,table]{beamer}
\usepackage{../../slides,../../math}
\definecolor{accent}{HTML}{2B5269}
\definecolor{accent2}{HTML}{9D2235}

\title{Topic 6: Fixed Effects, Difference-in-differences, and Factor Models}
\subtitle{\it  ECON 5783 — University of Arkansas}
\date{Fall 2024}
\author{Prof. Kyle Butts}

\usepackage{hhline} % for DID tables

\begin{document}

% ------------------------------------------------------------------------------
\begin{frame}[noframenumbering,plain]
\maketitle

% \bottomleft{\footnotesize $^*$A bit of extra info here. Add an asterich to title or author}
\end{frame}
% ------------------------------------------------------------------------------

\begin{frame}{Imputation Estimator review}
  The last set of slides, we introduced an ``imputation estimator'' for panel data treatment effects:

  \begin{enumerate}
    \item Estimate model for $y_{it}(\infty)$ using observations with $d_{it} = 0$ and get fitted values for full sample, $\hat{y}_{it}(\infty)$
    
    \item Regress $y_{it} - \hat{y}_{it}(\infty)$ on $d_{it}$ or event-study indicatorsto estimate treatment effects
    \begin{itemize}
      \item Estimating the overall effect, $\text{ATT}$, or dynamic effects of being treated for $\ell$ periods, $\text{ATT}^\ell$ respectively
    \end{itemize}
  \end{enumerate}
  
  \pause
  \bigskip
  This topic will extend this procedure to ``factor models'' that will allow more general trending behavior
\end{frame}

\section{Factor Models}

\begin{frame}{Factor Model}
  Untreated potential outcomes are given by a factor model:
  $$
    y_{it}(0) = \sum_{r = 1}^{p} \purple{f_{t,r}} * \orange{\gamma_{i,r}} + u_{it}
  $$

  \begin{itemize}
    \item $f_{t, r}$ is the $r$-th \textbf{\color{purple} factor} (macroeconomic shock) at time $t$.
    \item $\gamma_{i,r}$ is unit i's \textbf{\color{orange} factor loading} (exposure) to the $r$-th factor.
  \end{itemize}
\end{frame}

\begin{frame}{Factor Model Example}
  \vspace*{-\bigskipamount}
  $$
    y_{it}(0) = \sum_{r = 1}^{p} \purple{f_{t,r}} * \orange{\gamma_{i,r}} + u_{it}
  $$

  \bigskip
  If we are thinking about housing prices, $y_{it}$:
  \begin{itemize}
    \item $\orange{\gamma_i}$ are characteristics of neighborhood / house
    \item $\purple{f_{t}}$ are demand shocks in each period
  \end{itemize}
\end{frame}

\begin{frame}{Factor Model Example}
  \vspace*{-\bigskipamount}
  $$
    y_{it}(0) = \sum_{r = 1}^{p} \purple{f_{t,r}} * \orange{\gamma_{i,r}} + u_{it}
  $$

  \bigskip
  If we are thinking about wages, $y_{it}$:
  \begin{itemize}
    \item $\orange{\gamma_i}$ are worker's latent skills (e.g. computer skills)
    \item $\purple{f_{t}}$ reflect changing firm's sdemand for skills
  \end{itemize}
\end{frame}

\begin{frame}{Factor Model Example}
  \vspace*{-\bigskipamount}
  $$
    y_{it}(0) = \sum_{r = 1}^{p} \purple{f_{t,r}} * \orange{\gamma_{i,r}} + u_{it}
  $$

  \bigskip
  If we are thinking about county employment, $y_{it}$:
  \begin{itemize}
    \item $\orange{\gamma_i}$ are characteristics of a county (e.g. their manufacturing share)
    \item $\purple{f_{t}}$ reflect national shocks to the economy (e.g. the ``China shock'')
  \end{itemize}
\end{frame}

\begin{frame}{Two-way Fixed Effect vs. Factor Model}
  The factor model is a generalization of the TWFE model. If $\bm{f}_{t} = (\lambda_t, 1)'$ and $\bm{\gamma}_i = (1, \mu_i)'$, then our factor model becomes the TWFE model:
  $$
    y_{it}(0) = \bm{f}_t' \bm{\gamma}_i + u_{it} = \lambda_t + \mu_i + u_{it}
  $$
  
  \bigskip
  We can add unit and/or time fixed-effects as `known' factors if we want
\end{frame}


\begin{frame}{Factor Model and Parallel Trends}
  Say you have a single treatment timing and two periods. Let $D_i$ be out treated group indicator. Then
  \begin{align*}
    \expec{\Delta y_{i}}{D_i = d} 
    &= \expec{y_{i1} - y_{i0}}{D_i = d} \\ 
    &= \Delta \bm{f} \expec{\bm{\gamma}_i}{D_i = d} 
  \end{align*}
  
  \bigskip
  Under a factor model, the average change in $y_{it}$ for group $D_i = d$ is the change in factor shocks $\bm{f}$ times the average exposure to those shocks 
\end{frame}

\begin{frame}{Factor Model and Parallel Trends}
  \vspace*{-\bigskipamount}
  $$ 
    \expec{\Delta y_{i}}{D_i = d} = \Delta \bm{f} \expec{\bm{\gamma}_i}{D_i = d} 
  $$
  
  Say the treated group has higher exposure to a shock than the control group
  \begin{itemize}
    \item $\implies$ the trends differ by treatment status
  \end{itemize}

  \pause
  \bigskip
  That is, a factor model allows for ``non-paralell trends'' based on difference in exposures to shocks
\end{frame}

\begin{frame}{Example}
  \vspace*{-\bigskipamount}
  $$
    y_{it}(0) = \sum_{r = 1}^{p} \purple{f_{t,r}} * \orange{\gamma_{i,r}} + u_{it}
  $$

  \bigskip
  Say we are thinking about neighborhood housing prices, $y_{nt}$. We are interested in some treatment $D_n$, e.g. access to subways. 
  \begin{itemize}
    \item Say $\orange{\gamma_n}$ is the walkability of the neighborhood
    \item $\purple{f_{t}}$ are demand shocks for walkable neigbhorhoods
  \end{itemize}

  \bigskip
  If new subways are built in more walkable neighborhoods, then we do not believe parallel trends hold in this setting
\end{frame}







\end{document}
