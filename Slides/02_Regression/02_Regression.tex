\documentclass[aspectratio=169,t,11pt,table]{beamer}
\usepackage{../../slides,../../math}
\definecolor{accent}{HTML}{2B5269}
\definecolor{accent2}{HTML}{9D2235}

\title{Topic 2: Regression Toolkit}
\subtitle{\it  ECON 5783 — University of Arkansas}
\date{Fall 2025}
\author{Prof. Kyle Butts}

\begin{document}

% ------------------------------------------------------------------------------
\begin{frame}[noframenumbering,plain]
\maketitle

% \bottomleft{\footnotesize $^*$A bit of extra info here. Add an asterich to title or author}
\end{frame}
% ------------------------------------------------------------------------------

\begin{frame}{Regression Toolkit}
  We learn:
  \begin{itemize}
    \item What is the ``Conditional Expectation Function (CEF)''
    \item Curse of dimensionality and why we need models
    \item Modelling choices
    \item What is Omitted Variable Bias and why is it useful for thinking of causal effects
    \item Deeper understanding of regression: FWL Theorem
  \end{itemize}

  \bigskip
  Readings to complement this lecture are:
  \begin{itemize}
    \item Angrist and Pischke (2009) \emph{Mostly Harmless Econometrics}, Chatper 3: Intro, section 3.1, and section 3.2 
  \end{itemize}
\end{frame}

\begin{frame}{Linear Regression Bootcamp}
  This set of slides will serve as a `bootcamp' into one of the most popular tools in the applied researcher's toolkit: linear regression
  \begin{itemize}
    \item Creates a simple and interpretable model of $y$
    
    \item Has desirable properties for causal inference even if the outcome is not linear in covariates
  \end{itemize}
\end{frame}

\section{Conditional Expectation Function}

\subsection{Conditional Expectation Function}

\begin{frame}{The Conditional Expectation Function}
  In particular, we will think a lot about the \alert{Conditional Expectation Function} (CEF) of $y_i$ given $\bm{X}_i = (X_{i1}, \dots, X_{ip})'$:
  $$
    g(\bm{x}) \equiv \expec{y_i}{\bm{X}_i = \bm{x}}
  $$
  \begin{itemize}
    \item This reads ``$g(\bm{x})$ is the expected value of $y_i$ conditional on the unit having $\bm{X}_i = \bm{x}$''
  \end{itemize}

  \pause
  \bigskip
  The easiest way to estimate this for a given $\bm{x}$ is to average $y_i$ for units with $\bm{X}_i = \bm{x}$. 
  \pause
  \begin{itemize}
    \item Only uses observations with $\bm{X}_i = \bm{x}$ (or $\bm{X}_i \approx \bm{x}$ when $\bm{X}_i$ is continuous), so that is the relavent `$n$' when considering sample size
  \end{itemize}
\end{frame}

\begin{frame}{Estimation Procedure for CEF}
  For each value of $\bm{x}$, 
  \begin{itemize}
    \item Subset population to only the units with $\bm{X}_i = \bm{x}$
    
    \item Average value of $y_i$ for those units
  \end{itemize}
  
  This is the value $g(\bm{x})$ for that given $\bm{x}$
\end{frame}

\begin{frame}{Example with Discrete Variable}
  Say, we have $w_i$ as wages and $D_i$ is an indicator for college attendance. Then, we can estimate the CEF of wages conditional on college attendance as

  $$
    g(1) = \expec{y_i}{D_i = 1} \quad\text{ and }\quad g(0) = \expec{y_i}{D_i = 0}
  $$

  \bigskip
  This is just just the average wage for college attendees and non-attendees
\end{frame}

\begin{frame}{Regression Framework}
  To estimate these given a sample of workers, we could regress
  $$w_i = \alpha + D_i \tau + u_i$$
  
  \begin{itemize}
    \item $\hat{\alpha}$ is our estimate of $g(0)$ and $\hat{\alpha} + \hat{\tau}$ is our estimate of $g(1)$
  \end{itemize}
\end{frame}

\begin{frame}{Example with two discrete variables}
  Now, we have $w_i$ as wages and $\bm{X}_i$ consists of an indicator for college attendence, $D_i$, and $F_i$ is an indicator for being a female. 
  
  \bigskip
  Now we have four distinct values of $\bm{X}_i$, $(D_i, F_i) \in \{(0, 0), (1, 0), (0, 1), (0, 0)\}$. 
  \begin{itemize}
    \item Estimating the CEF would consist of just four sub-sample averages (e.g. female college attendees)
  \end{itemize}
\end{frame}

\begin{frame}{Regression Framework}
  Note, it is \emph{not} enough to regress wages on an indicator for college and an indicator for female. 
  We need to include an interaction term as well! 

  $$w_i = \alpha + D_i \beta_1 + F_i \beta_2 + (D_i \cdot F_i) \beta_3 + u_i$$
  
  \bigskip
  \begin{itemize}
    \item Without the interaction term, we are assuming the effect of college is the same for female and male workers
  \end{itemize}
\end{frame}

\begin{frame}{Example with single continuous variable}
  If we move from a discrete to a single continuous $X_i$, we can still estimate the CEF by averaging $y_i$ for individuals with $X_i \approx x$
  \begin{itemize}
    \item Only now, we have a problem where there are infinitely many values of $x$ and we have to decide how close is `close enough' to $x$
  \end{itemize}

  \bigskip
  We will estimate by making many small bins of $x$ along the full range of $X$. 
  \begin{itemize}
    \item This is called `non-parametric' estimation in the statistics literature
  \end{itemize}
\end{frame}

\imageframe{figures/f_examples_plot_dgp.pdf}
\imageframe{figures/f_examples_cef.pdf}

\begin{frame}{Single continuous and single discrete variable}
  If we have a continuous $X_{1i}$ and a discrete $D_i$ variable, we can do:
  \begin{itemize}
    \item Average value of $y$ for bins of $X_i$ \emph{separately} for $D_i = 0$ and $D_i = 1$. 
  \end{itemize}
\end{frame}

\begin{frame}{Estimation of the CEF}
  As we discussed before, we could estimate $g(x) \equiv \expec{y_i}{\bm{X}_i = x}$ by averaging over individuals with $\bm{X}_i = x$
  \begin{itemize}
    \item In the case where $\bm{X}_i$ is a discrete variable taking values $x_1, \dots, x_L$, this is just sub-sample averages for $\bm{X}_i = x_\ell$
    
    \item With $\bm{X}_i$ as a continuous variable, we do a bunch of `binned' averages.
  \end{itemize}

  \bigskip
  As we start adding variables, we have to start `interacting' the variables, creating many many different sub-samples
\end{frame}

\begin{frame}{The ``curse of dimensionality''}
  When $\bm{X}_i$ is a multi-dimensional vector with many continuous variables, we end up with \emph{a lot} of subsamples we want to take averages of. 

  \bigskip
  The density around any particular value $\bm{x}$ is typically going to be small or near-zero 
  \begin{itemize}
    \item In samples, our estimates will be very noisy or even impossible to calculate for given values of $\bm{x}$ 
  \end{itemize}
\end{frame}

\begin{frame}{The Conditional Expectation Function}
  Uses of the conditional expectation function:
  \begin{enumerate}
    \item \alert{Descriptive}: how $y$ on average changes as $X$ changes 
    \begin{itemize}
      \item By definition, compare $g(x_1)$ to $g(x_2)$
    \end{itemize} 

    \bigskip
    \item \alert{Prediction}: if we know $\bm{X}_i$, our best guess for $y_i$ is $g(\bm{X}_i)$ 
    
    \bigskip
    \item \alert{Causal inference}: what happens to $y_i$ if we \emph{manipulate} $\bm{X}_i$ 
    \begin{itemize}
      \item Sometimes, the point of this class is to describe when! 
    \end{itemize}
  \end{enumerate}
\end{frame}


\begin{frame}{Preview of Conditional Expectation Function usages}
  One main reason why we care about modeling $Y$ is because causal inference is a missing data problem
  \begin{itemize}
    \item For the treated units, we do not observe what their outcomes would be in the absence of treatment, $Y(0)$
    \item For the control units, we do not observe their $Y(1)$
  \end{itemize}

  \pause
  \bigskip
  If we fit a model for $\expec{Y_i(0)}{\bm{X}_i = x}$, we can use this to make predictions for the treated units 
  
  \pause
  \begin{itemize}
    \item The model predicting out of sample for our treated group requires certain conditions discussed in topic 3
  \end{itemize}
\end{frame}

\section{Modeling the CEF}


% \begin{frame}{Prediction Error and the CEF}
%   The prediction error of the conditional expectation function is given by $\varepsilon_i = y_i - g(\bm{X}_i)$. For any $x$, we have 
%   \begin{align*}
%     \expec{\varepsilon_i}{\bm{X}_i = x} 
%     &= \expec{y_i - \expec{y_i}{\bm{X}_i = x}}{\bm{X}_i = x} \\
%     &= \expec{y_i}{\bm{X}_i = x} - \expec{y_i}{\bm{X}_i = x} \\
%     &= 0
%   \end{align*}
%   
%   \pause
%   The prediction error is unpredictable given $\bm{X}_i = x$
%   \begin{itemize}
%     \item We have \emph{used up} all the information that $\bm{X}_i$ can give us. 
%     \item This is not true for general $f(X)$
%   \end{itemize}
% \end{frame}
% 
% \begin{frame}{Mean-square prediction error}
%   To provide a summary measure of fit, we want a `average' prediction error over the population
%   \begin{itemize}
%     \item If we took the average of prediction error, positive and negative prediction errors would cancel out
%   \end{itemize}
% 
%   \pause
%   \bigskip
%   The \alert{mean-square (prediction) error} (MSE) for some model $f$ is calculated as:
%   \begin{equation}\label{eq:mspe}
%     \text{MSE}(f) \equiv \expec{\left( y_i - f(\bm{X}_i) \right)^2}
%   \end{equation}
%   \vspace*{-\bigskipamount}
%   \begin{itemize}
%     \item Averaged over the population
%   \end{itemize}
% \end{frame}
% 
% \begin{frame}{Best predictor of $y$}
%   If you have data on \emph{the full population}, the best possible model for predicting $y$ in terms of minimizing the mean-square prediction error is the conditional expectation function
%   \begin{itemize}
%     \item Proof on the next three slides for those interested
%   \end{itemize}
% \end{frame}
% 
% \begin{frame}{Proof: Optimal model for $y$}
%   The model $f$ that minimizes the mean-square prediction error is the conditional expectation function.
%   \begin{align*}
%     &\expec{\left( y_i - f(\bm{X}_i) \right)^2} = \expec{\left( y_i - g(\bm{X}_i) + g(\bm{X}_i) - f(\bm{X}_i) \right)^2} \\ \pause
%     &\quad= \expec{\left( y_i - g(\bm{X}_i) \right)^2} + \expec{\left( g(\bm{X}_i) - f(\bm{X}_i) \right)^2} + 2 \expec{\left( y_i - g(\bm{X}_i) \right) \left( f(\bm{X}_i) - g(\bm{X}_i) \right)}
%   \end{align*}
% 
%   \begin{itemize}
%     \item The first term does not depend on $f$
%   \end{itemize}
% \end{frame}
% 
% \begin{frame}{Proof: Optimal model for $y$}
%   The last term equals $0$: 
%   \begin{align*}
%     & \expec{\left( y_i - g(\bm{X}_i) \right) \left( f(\bm{X}_i) - g(\bm{X}_i) \right)} \\
%     &\quad= \expec{\expec{\left( y_i - g(\bm{X}_i) \right) \left( f(\bm{X}_i) - g(\bm{X}_i) \right)}{\bm{X}_i}} \\
%     &\quad= \expec{\left( \expec{y_i}{\bm{X}_i} - g(\bm{X}_i) \right) \left( f(\bm{X}_i) - g(\bm{X}_i) \right)} \\
%     &\quad= \expec{\left( g(\bm{X}_i) - g(\bm{X}_i) \right) \left( f(\bm{X}_i) - g(\bm{X}_i) \right)} \\
%     &\quad= 0
%   \end{align*}
% \end{frame}
% 
% \begin{frame}{Proof: Optimal model for $y$}
%   The model $f$ that minimizes the mean-square prediction error is the conditional expectation function.
%   \begin{align*}
%     &\argmin_{f} \expec{\left( y_i - f(\bm{X}_i) \right)^2} = \expec{\left( y_i - g(\bm{X}_i) + g(\bm{X}_i) - f(\bm{X}_i) \right)^2} \\
%     &\quad= \argmin_{f} \expec{\left( y_i - g(\bm{X}_i) \right)^2} + \expec{\left( g(\bm{X}_i) - f(\bm{X}_i) \right)^2} + 0
%   \end{align*}
% 
%   \begin{itemize}
%     \item Minimizing this with respect to $f$ only involves the second term so we set $f(\bm{X}_i) = g(\bm{X}_i)$
%   \end{itemize}
% \end{frame}
% 
% \begin{frame}{Optimal model for $y$}
%   Therefore, in terms of mean-square prediction error, the conditional expectation function is the best predictor of $y$
% \end{frame}

\begin{frame}{Modelling the CEF}
  We have an outcome variable $y$ and a set of $p$ different predictor variables $X = (X_1, X_2, \dots, X_p)$. 
  \begin{itemize}
    \item For some observations we observe both $X$ and $y$; this is essential to \alert{fit} a model
  \end{itemize}

  \bigskip
  We can write our model in a general form as
  $$
    y = f(X) + \varepsilon,
  $$
  where $f$ is some unknown (but fixed) function of $X$. 
\end{frame}

\begin{frame}{Estimation of the CEF}
  As we discussed before, we could estimate $g(\bm{x}) \equiv \expec{y_i}{\bm{X}_i = \bm{x}}$ by averaging over individuals with $\bm{X}_i = \bm{x}$

  \bigskip
  But as $\bm{X}_i$ has more variables, ``curse of dimensionality'' makes this procedure infeasible
  \begin{itemize}
    \item We need to make assumptions on the shape of $g(x)$ to make estimation feasible
  \end{itemize}
\end{frame}


\subsection{Linear Model of Conditional Expectation Function}

\begin{frame}{Linear Model}
  It is common to propose a \emph{parametric} model of the conditional expectation function:
  $$
    y_i = 
    \bm{X}_i' \beta + \text{error} = 
    \sum_{k=1}^p X_{i,k} \beta_k + \text{error}
  $$
  \begin{itemize}
    \item We model $y$ as a linear function of the covariates

    \item Assume that one of the variables in $\bm{X}_i$ is an intercept term
  \end{itemize}
\end{frame}

\begin{frame}{Fitting via Ordinary Least Squares}
  ``Fitting'' the linear model involves selecting $\beta$ to make out model the ``best'' linear predictor of $y$:
  $$
    \hat{\beta}_{\texttt{OLS}} \equiv \argmin_{\beta} \expec{\left( y_i - \bm{X}_i' \beta \right)^2}
  $$

  \bigskip\bigskip
  We can optimize this by taking first-order conditions and set equal to zero:
  \begin{align*}
    &\expec{ \bm{X}_i \left( y_i - \bm{X}_i' \beta_{\texttt{OLS}} \right) } = 0 \\
    &\implies \expec{ \bm{X}_i y_i } - \expec{\bm{X}_i \bm{X}_i'} \beta_{\texttt{OLS}} = 0 \\
    &\implies \beta_{\texttt{OLS}} = (\expec{\bm{X}_i \bm{X}_i'})^{-1} \expec{ \bm{X}_i y_i }
  \end{align*}
\end{frame}

\begin{frame}{Ordinary Least Squares Estimator}
  We can estimate using a sample of observations:
  $$
    \hat{\beta}_{\texttt{OLS}} = \left( \sum_{i=1}^n \bm{X}_i \bm{X}_i' \right)^{-1} \sum_{i=1}^n \bm{X}_i y_i
  $$

  \bigskip
  Or in matrix notation
  $$
    \hat{\beta}_{\texttt{OLS}} = (\bm{X}' \bm{X})^{-1} \bm{X}' \bm{y}
  $$
  \begin{itemize}
    \item $\bm{X}$ is the $n \times k$ matrix with row given by $\bm{X}_i'$ and $\bm{y}$ is the column vector of outcome variables
  \end{itemize}
\end{frame}

\begin{frame}{Linearity}
  When is a linear model of $g(x) \equiv \expec{y_i}{\bm{X}_i = \bm{x}}$ a good assumption?
  \begin{itemize}
    \item In some cases, the data might look to grow linearly in each $X_{i,k}$, in which case, it is a reasonable assumption
  \end{itemize}

  \bigskip
  But many times, the linear model will fall short. For example:
  \begin{itemize}
    \item wages seem to grow quadratically in age
    
    \medskip
    \item Square footage of homes seems to have diminishing returns to price

    \medskip
    \item Returns to work experience depend on whether you have a college degree (interaction)
  \end{itemize}
\end{frame}

\begin{frame}{``Extending'' linear models}
  We can, of course, increase the performance of our model by doing things like polynomials of variables and interaction between terms.

  \bigskip
  It becomes necessary to distinguish between the variables you are using, $\bm{X}_i$ and the terms you include in your model:
  $$
    \bm{W}_i = \left( g_1(\bm{X}_i), \dots, g_K(\bm{X}_i) \right)'
  $$

  \medskip
  \begin{itemize}
    \item When $g_1(\bm{X}_i) = x_{1, i}, \dots, g_p(\bm{X}_i) = X_{i, p}$ we end up with the original model
    
    \item But, of course, we can include polynomials and/or interactions as well
  \end{itemize}
\end{frame}

\begin{frame}{A `correctly specified' model}
  We say a \emph{linear model} is \alert{correctly specified} if the CEF is exactly equal to the model we are estimating:
  $$
    g(\bm{x}) \equiv \expec{Y_i}{\bm{X}_i = \bm{x}}= \bm{w} \gamma_0
  $$

  \bigskip
  That is, the true conditional expectation is linear our set of terms $\bm{W}_i$
  \begin{itemize}
    \item We have `fully used' the information in $\bm{X}_i$ with our terms $\bm{W}_i$
  \end{itemize}
\end{frame}

\begin{frame}{Example: Discrete variables}
  When $\bm{X}_i$ is a discrete variable taking values $x_1, \dots, x_L$, consider a linear model consisting of a set of \alert{indicator variables} for each value of $x_\ell$:
  \begin{equation}\label{eq:dummy_variable_regression}
    y_i = \sum_{\ell = 1}^L \one{\bm{X}_i = x_\ell} \beta_\ell + u_i
  \end{equation}
  \pause
  \begin{itemize}
    \item The ordinary least-squares estimator estimates $\hat{\beta}_{\ell} = \expechat{y_i}{\bm{X}_i = x_\ell}$

    \pause
    \item In this case, the CEF is \emph{correctly specified} as the linear model (\ref{eq:dummy_variable_regression}).
  \end{itemize}
\end{frame}

\begin{frame}{Example: Discrete variables}
  As mentioned before, if we have multiple discrete variables, we need to include interaction terms as well to ensure we have a correctly specified model
  \begin{itemize}
    \item If we do not include interaction terms, we need the true coefficients on the interactions to be 0!
  \end{itemize}
\end{frame}

\begin{frame}{Ommitted Categories}
  When we include a constant in the regression (or have multiple sets of indicator variables) we have issues of \alert{multi-collinearity}:
  $$
    y_i = \alpha + \sum_{\ell = 2}^L \one{\bm{X}_i = x_\ell} \beta_\ell + u_i
  $$

  \bigskip
  We need to drop (at least) one of the indicator variables (say $\one{\bm{X}_i = x_1}$). This serves as the ``reference category''
  $$
    \hat{\beta}_{\ell} = \expechat{y_i}{\bm{X}_i = x_\ell} - \expechat{y_i}{\bm{X}_i = x_1}
  $$
  
  $\hat{\beta}_{\ell}$ is the mean of group $\ell$ relative to the omitted group
\end{frame}

\subsection{Making models more flexible}

\begin{frame}{Prediction model}
  We have an outcome variable $y$ and a set of $p$ different predictor variables $X = (X_1, X_2, \dots, X_p)$. 
  \begin{itemize}
    \item For some observations we observe both $X$ and $y$; this is essential to \alert{fit} the model
  \end{itemize}

  \bigskip
  We can write the model in a general form as
  $$
    y = f(X) + \varepsilon,
  $$
  where $f$ is some unknown (but fixed) function of $X$. By definition $\varepsilon \equiv y - f(X)$ is the \alert{error term} that is needed to fit the data perfectly
\end{frame}

\begin{frame}{Prediction \emph{model}}
  There are many different possible models of $f$ ranging from a linear model; a `smooth' model (polynomial or other); or a fully non-parametric function
\end{frame}

\imageframe{figures/f_examples_plot_raw.pdf}
\imageframe{figures/f_examples_plot_pred_1.pdf}
\imageframe{figures/f_examples_plot_pred_2.pdf}
\imageframe{figures/f_examples_plot_pred_3.pdf}
\imageframe{figures/f_examples_plot_pred_4.pdf}

\begin{frame}{Prediction model}
  There are many different possible models of $f$ ranging from a linear model; a `smooth' model (polynomial or other); or a fully non-parametric function
  
  \bigskip
  The more `fancy' a model:
  \begin{itemize}
    \item The more \alert{flexible} the relationship between $y$ and $X$ can be
    
    \item The larger the risk of \alert{overfitting} the data
    
    \item The less \alert{interpretable} the model becomes
  \end{itemize}
\end{frame}

\begin{frame}{Flexibility vs. Overfitting}
  \vspace{-\bigskipamount}
  \begin{center}
    \includegraphics[width = \textwidth]{figures/f_examples_plot_dgp.pdf}
  \end{center}
\end{frame}

\begin{frame}{Flexibility vs. Overfitting}
  \vspace{-\bigskipamount}
  \begin{center}
    \includegraphics[width = \textwidth]{figures/f_examples_overfitting.pdf}
  \end{center}
\end{frame}

\begin{frame}{Flexibility vs. Overfitting}
  By making the model more and more \emph{flexible}, you risk overfitting more and more

  \begin{itemize}
    \item A solution is to evaluate your model fit using outside `testing data' (hold out some observations from fitting the model)
  \end{itemize}

  \pause
  \bigskip
  This technique is not as common when you care more about the associations between variables (interpreting the model)
  \begin{itemize}
    \item Not really a good reason other than "that is more complicated"
  \end{itemize}
\end{frame}

\subsection{More Flexible Approximations (\texttt{binscatter})}

\begin{frame}{Partially linear model}
  The \alert{Partially linear model} mixes high model flexibility in a key variable we care about and linear model for the rest of the covariates:
  $$
    y_i = \mu(X_i) + \bm{Z}_i' \beta + u_i
  $$
  \begin{itemize}
    \item $\mu(X_i)$ is a highly flexible function

    \item $\bm{Z}_i'$ is a set of \emph{linear} control variables
  \end{itemize}

  \bigskip
  This allows you to prevent the curse of dimensionality by linearly controlling for most of the variables. 
  Allows a flexible model for the key variable of interest, $\bm{X}_i$, that is good for graphing
\end{frame}

\begin{frame}{Partially linear model}
  $$
    y_i = \mu(X_i) + \bm{Z}_i' \beta + u_i
  $$
  
  \bigskip
  One recent way of estimating this is using a `binscatter' regression 
  \begin{itemize}
    \item Popularized by Raj Chetty and coauthors since they had millions of observations (too many for normal scatterplots)
  \end{itemize}
\end{frame}

\begin{frame}{Binscatter Regression}
  $$
    y_i = \mu(X_i) + \bm{Z}_i' \beta + u_i
  $$
  One recent way of estimating this is using a `binscatter' regression:
  \begin{enumerate}
    \item Chop $X_i$ variable into $J$ bins with an equal number of observations into each bin
    
    \item Fit some polynomial of $X_i$ just within each bin (interact $X_i$ polynomial with bin indicators)
  \end{enumerate}
\end{frame}

\imageframe{figures/ex_binsreg_raw.pdf}
\imageframe{figures/ex_binsreg_split_into_bins.pdf}
\imageframe{figures/ex_binsreg_bins.pdf}
\imageframe{figures/ex_binsreg_bins_add_linear.pdf}
\imageframe{figures/ex_binsreg_bins_add_smooth.pdf}



\section{Omitted Variable Bias (OVB)}

\begin{frame}{Difference between true model and model we estimate}
  Say there is a true causal model for $y$
  $$
    y_i = \beta_0 + X_{i1} \beta_1 + X_{i2} \beta_2 + \varepsilon_i
  $$
  \begin{itemize}
    \item Assume $\expec{\varepsilon_i}{\bm{X}_i} = 0$ so that $\beta_1$ is the true causal effect
  \end{itemize}
  
  \pause
  \bigskip
  But we only estimate a `short' regression specification
  $$
    y_i = \delta_0 + X_{i1} \delta_1 + error
  $$

  \bigskip
  What is the relationship between $\beta_1$ the true causal effect and the coefficient $\delta_1$?
\end{frame}

\begin{frame}{Omitted Variable Bias}
  \vspace*{-\bigskipamount}
  $$
    \underbrace{y_i = \beta_0 + X_{i1} \beta_1 + X_{i2} \beta_2 + \varepsilon_i}_{\text{``long regression''}} \quad\text{ and }\quad \underbrace{y_i = \delta_0 + X_{i1} \delta_1 + \text{error}_i}_{\text{``short regression''}}
  $$

  \bigskip
  We have the following relationship:
  \begin{align*}
    \delta_1 
    &= \frac{\cov(X_1, y)}{\var{X_1}} \\
    &= \frac{\cov(X_1, \beta_0 + X_{1} \beta_1 + X_{2} \beta_2 + \varepsilon)}{\var{X_1}} \\ \pause
    &= \beta_1 + \beta_2 \frac{\cov{X_1, X_2}}{\var{X_1}} 
  \end{align*}
\end{frame}

\begin{frame}{Omitted Variable Bias}
  $$
    \hat{\delta}_1 = \beta_1 + \beta_2 \frac{\cov{X_1, X_2}}{\var{X_1}} 
  $$

  \bigskip
  The reason this is true is due to regression being a prediction model!
  \begin{itemize}
    \item If $X_1$ and $X_2$ are correlated, then knowing about $X_1$ tells me information on $X_2$
    
    \item I would want to use that implicit information on $X_2$ to predict $y$ as well! 
  \end{itemize}

  \bigskip
  $\implies$ take the effect of $\beta_2$ times how I think $X_1$ tells me about $X_2$
\end{frame}

\begin{frame}{Omitted Variable Bias}
  $$
    \hat{\delta}_1 = \beta_1 + \beta_2 \frac{\cov{X_1, X_2}}{\var{X_1}} 
  $$

  \bigskip
  We can often times `sign' the bias:
  \begin{itemize}
    \item The sign of $\beta_2$ is what we think the effect of $X_2$ is on $y$
    \item $\cov{X_1, X_2}$ is how $X_1$ and $X_2$ are related in the population
  \end{itemize}
\end{frame}

\begin{frame}{Signing the Bias}
  \begin{center}
    \begin{tabular}{@{\extracolsep{5pt}} l | c | c | c}
      \toprule
                    & $\cov(X_1, X_2) > 0$ & $\cov(X_1, X_2) < 0$ & $\cov(X_1, X_2) = 0$ \\
      \midrule
      $\beta_2 > 0$ & positive bias        & negative bias  & no bias\\
      \midrule
      $\beta_2 < 0$ & negative bias        & positive bias  & no bias\\
      \midrule
      $\beta_2 = 0$ & no bias              & no bias        & no bias\\

      \bottomrule
    \end{tabular}
  \end{center}

  \bigskip
  If $X_2$ is unrelated to $X_1$ \emph{or} $X_2$ has no effect on $y$, then we have no problem
\end{frame}

\subsection{Reinterpreting selection bias as OVB}

\begin{frame}{Omitted Variable Bias}
  Let $X_1$ is an indicator variable, call it $D$. 
  \begin{align*}
    \cov{D, X_2} 
    &= \expec{ (D - \expec{D}) (X_2 - \expec{X_2})} \\
    &= \expec{D (X_2 - \expec{X_2})} \\
    &= \pi \expec{X_2}{D = 1} - \pi \expec{X_2}
  \end{align*}

  Let $\pi = \prob{D = 1}$ and note from definition, $\var{D} = \pi (1 - \pi)$. Then,
  $$
    \delta_1 = \beta_1 + \frac{\beta_2}{(1 - \pi)} \left( \expec{X_2}{D = 1} - \expec{X_2} \right)
  $$
\end{frame}

\begin{frame}{Selection Bias}
  \vspace*{-\bigskipamount}
  $$
    \delta_1 = \beta_1 + \frac{\beta_2}{(1 - \pi)} \left( \expec{X_2}{D = 1} - \expec{X_2} \right)
  $$

  \bigskip
  In our context of $D$ being a treatment indicator, $\delta_1$ is our treatment effect estimate and $\beta_1$ is the true ATT.

  \pause
  \bigskip
  We see that if the mean of $X_2$ differs for the treatment group, then our estimate is biased
  \begin{itemize}
    \item E.g. if $D$ is college attendance and $X_2$ is parental income, then our treatment effect is biased if college attendees have difference average parental income
  \end{itemize}
\end{frame}


\begin{frame}{OVB In Practice}
  A lot of research will run regressions that look like 
  $$
    y_i = D_i \tau + \bm{X}_i' \beta + \varepsilon_i
  $$

  \bigskip
  The \emph{key things} you will want to do is think through what might show up in the error term
  \begin{enumerate}
    \item If those omitted variables are correlated with $D_i$ (after controlling for $\bm{X}_i$) and have an effect on $y_i$, then you have problems interpreting the effect as causal
  \end{enumerate}
\end{frame}


\section{Frisch-Waugh-Lovell Theorem}

\begin{frame}{Projection Matrix}
  Before we describe the Frisch-Waugh-Lovell theorem, let's define a few terms. Consider our regression estimator
  $$
    \hat{\beta} = \left( X'X \right)^{-1} X' y
  $$
  
  \bigskip
  We could then create fitted values by multiplying $X$ by our coefficient of interest: 
  $$
    X \hat{\beta} = X \left( X'X \right)^{-1} X' y \equiv P_X y
  $$
  \begin{itemize}
    \item We define the \alert{Projection Matrix} as $P_X$ to be the fitted values from a regression of a variable on the variables $X$.
  \end{itemize}
\end{frame}

\begin{frame}{Residuals}
  The residuals from the regression are given by $\hat{\varepsilon} = y - \hat{y} = y - P_X y$

  \bigskip
  In matrix notation, we can write this as $\hat{\varepsilon} = (I - P_X) y$. We define $M_X$ to be the \alert{annihilator matrix} with $M_X \equiv I - P_X$

  \pause
  \begin{itemize}
    \item The annihilator matrix first predicts $y$ using a linear model of $X$ and then subtracts off the prediction
  \end{itemize}
\end{frame}

\begin{frame}{Residuals}
  From regression algebra we have the residuals are (linearly) uncorrelated with $\bm{X}_i$ 
  $$
    \expec{\bm{X}_i \hat{\varepsilon}_i} = 0
  $$
  
  \bigskip
  \pause 
  If we assume that the CEF $\expec{y_i}{\bm{X}_i} = \bm{X}_i' \beta$, then we can go further and say 
  $$
    \expec{\hat{\varepsilon}_i}{\bm{X}_i = x} = 0
  $$
  \begin{itemize}
    \item the remaining variation in $y_i$, given by $\hat{\varepsilon}_i$, is unpredictable given $\bm{X}_i$
  \end{itemize}
\end{frame}


\begin{frame}{Frisch-Waugh-Lovell Theorem}
  Consider the regression
  $$
    y_i = \tau D_i + W_i' \beta + u_i
  $$
  \begin{itemize}
    \item $D_i$ is a scalar variable of interest and $W_i$ is a $k \times 1$ vector of covariates
  \end{itemize}

  \bigskip
  We can of course estimate the regression coefficients $\hat{\tau}$ and $\hat{\beta}$ jointly in a single regression
\end{frame}

\begin{frame}{Frisch-Waugh-Lovell Theorem}
  The \alert{FWL theorem} shows that instead of doing one regression, we could estimate $\hat{\tau}_{\texttt{OLS}}$ by the series of steps:
  \begin{enumerate}
    \item Regress $y_i$ on $W_i$ and grab the residuals, $M_W y$
    \item Regress $D_i$ on $W_i$ and grab the residuals, $M_W D$
    \item Regress $M_W y$ on $M_W D$ to estimate $\hat{\tau}_{\texttt{FWL}}$
  \end{enumerate}

  \bigskip
  \pause
  The estimate $\hat{\tau}_{\texttt{FWL}}$ is going to be \emph{numerically identical} to $\hat{\tau}_{\texttt{OLS}}$.
  \begin{itemize}
    \item Up to degree-of-freedom correction, the standard errors will be identical as well (including robust and clustered standard errors)
    \begin{itemize}
      \item The final regression pretends we didn't estimate the $K$ coefficients on $W_i$
    \end{itemize}
  \end{itemize}
\end{frame}

\begin{frame}{Frisch-Waugh-Lovell Theorem}
  The FWL Theorem shows us how to think about the regression coefficient in a multivariate regression:
  \begin{itemize}
    \item We are predicting $D_i$ and $y_i$ using covariates $W_i$
    \item We are removing that predictable variation and seeing if the ``remaining variation'' in $y_i$ and $D_i$ are linearly correlated
  \end{itemize}

  \pause
  \bigskip
  To be clear, we do not have to run these regression; we can interpret our regression results as if we had run it using this procedure
\end{frame}

\begin{frame}{Example of Frisch-Waugh-Lovell Thinking}
  We want to know the causal effect of college on earnings
  \begin{itemize}
    \item $D_i$ is an indicator for a person going to college
    
    \item $y_i$ is the worker's earnings at age 25
    
    \item $W_i$ is a vector of covariates we think are important determinents of college attendance and/or earnings
  \end{itemize}

  Run this regression:
  $$
    y_i = \tau D_i + W_i' \beta + u_i
  $$

\end{frame}

\begin{frame}{Example of Frisch-Waugh-Lovell Thinking}
  The regression estimate will do the following:
  \begin{itemize}
    \item Predict whether a worker would go to college given the covariates $W_i$. The difference between $D_i$ and the prediction $\hat{D}_i$ is \emph{hopefully} due to random reasons
    
    \item Predict how those covariates $W_i$ would affect future earnings and remove that prediction. The remaining variation in wages is hopefully driven by (i) either college attendance, or (ii) other reasons that are uncorrelated with going to college
  \end{itemize}

  \bigskip
  It is important therefore to know a lot about your subject and know what causes treatment uptake $D_i$
\end{frame}

\begin{frame}{Example of Frisch-Waugh-Lovell Thinking}{College attendance}
  Like with omitted variable bias, this is a story of what variables did we not include. In our college attendance example, say $W_i$ is parental income and GPA. 
  \begin{itemize}
    \item Both are important drivers of college attendance, but not the only ones
  \end{itemize}

  \bigskip
  What are examples of other variables that can drive attendance?
\end{frame}



% \section{Binary Outcome Variable}



\end{document}
