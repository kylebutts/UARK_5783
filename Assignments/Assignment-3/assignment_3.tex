\documentclass[12pt]{article}
\usepackage{../../lecture_notes}
\usepackage{../../math}
\usepackage{../../uark_colors}

\hypersetup{
  colorlinks = true,
  allcolors = ozark_mountains,
  breaklinks = true,
  bookmarksopen = true
}

\begin{document}
\begin{center}
  {\Huge\bf Assignment \#3}
  
  \smallskip
  {\large\it  ECON 5783 — University of Arkansas}

  \medskip
  {\large Prof. Kyle Butts}
\end{center}

These assignments should be completed in groups of 1 or 2 but submitted individually. My preference is for you to use Rmarkdown files to have your code, results, and your answers to the questions intermixed. Since I am not requiring you to code in R for these assignments, you can use latex or microsoft word to write up answers alternatively. Not that in these cases, I would like you to upload your code seperately. 

\section*{Coding Exercise}

This assignment serves as an introduction to Monte-Carlo simulations and will assess your understanding of causal assumptions needed in different methods (today's is difference-in-means). Monte-Carlo simulations are a common strategy to assess the performance of estimators. The main idea is that you generate a dataset that either does or does not satisfy the underlying assumptions of a method and assess how the estimator does. 

In our case, we have the difference-in-means estiamtor which requires the independence assumptions, i.e. that the treated and untreated potential outcomes $Y_i(1)$ and $Y_i(0)$ are independent of treatment status $D_i$. 

In the file \texttt{break\_the\_simulation\_1.qmd}, there is a complete example of a monte carlo simulation. In it, we generate 1000 example datasets that satisfy the independence assumption and estimate the difference-in-means estimator. You can see how we run Monte Carlo simulations in the code and then display the estimates in a histogram. Note that the distribution of DIM estimates falls at the true treatment effect.

Your task is to break this simulation by modifying the \texttt{dgp\_broken} function. To break it, I want you to change the data generating process to \emph{break the identifying assumption} of the difference-in-means estimator. Please avoid doing silly things, e.g. changing the treatment effect so that the estimate is shifted.  


After doing so, describe what you did in the \texttt{\#\# Describe your thinking} section.


\end{document}
